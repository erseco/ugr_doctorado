\chapter{Antecedentes}

En este capítulo vamos analizar el estado de arte actual y las tecnologías candidatas a utilizarse en este proyecto en base a los objetivos que presentamos en el capítulo anterior.

\bigskip
Como ya vimos en la introducción, diversos estudios aseguran que es posible incrementar la seguridad de una configuración en base al uso de algoritmos genéticos, pero la mayoría son demostraciones teóricas sin ninguna implementación real cuantificable  \cite{john_evolutionary_2014} \cite{romero_sistema_2017} \cite{buji_genetic_2017}. 

\section{Herramientas existentes}
A pesar de las multiples aproximaciones teóricas \cite{schlenker_deceiving_2018} \cite{champagne_genetic_2018}, actualmente no existe ninguna herramienta, ni ningún ejemplo liberado de forma pública que permita comprobar y generar configuraciones e ir evolucionando las mismas para mejorar tanto su seguridad como su diversidad por lo que se ha optado por la realización de una herramienta capaz de realizar esto.

\bigskip
Para el análisis y cuantificación de vulnerabilidades si existe herramientas, además algunas de ellas liberadas con licencias abiertas. Hemos optado por OWASP ZAP ya que es uno de las herramientas de análisis de vulnerabilidades más utilizadas y además está liberada bajo la licencia GPLv3 \cite{free_software_foundation_gnu_2007}.

\section {Protocolos y/o servidores de red}

En esta sección analizaremos algunos de los protocolos de red más conocidos así como algunas de sus implementaciones. Ateniéndonos a la filosofía abierta de este proyecto nos limitaremos a las implementaciones libres de los mismos, además, diversas publicaciones demuestran que el software de código abierto es mas seguro que el software cerrado \cite{walia_comparative_2006} \cite{mansfield-devine_open_2008} \cite{clark_is_2009}.

\subsection {SMB - Server Message Block}

Es un protocolo de red desarrollado por IBM a principios de la década de los 90 y adoptado por Microsoft a partir de 1992. Este protocolo permite compartir archivos e impresoras en red.

\bigskip
Debido a la cantidad de sistemas compatibles es uno de los protocolos más utilizados para compartir ficheros en redes empresariales, esto hace que sea uno de los objetivos principales de muchos ciberataques.

\bigskip
Por poner un ejemplo, en mayo de 2017 hubo un ciberataque masivo causado por el ransomware WannaCry \cite{sarabia_mayor_2017}, dicho software hacía uso de un \textit{exploit} conocido como EternalBlue que conseguía penetrar en sistemas que todavía hacían uso de los protocolos SMB1 y SMB2 obligando a MicroSoft a publicar parches incluso para sistemas operativos que habían terminado su ciclo de vida (Windows XP y Windows Vista).


\subsubsection {Implementaciones libres}

\begin{table}[H]
\begin{tabular}{|l|l|}
\hline
Nombre                   & Samba                        \\ \hline
Licencia                 & GPLv3                        \\ \hline
Año de lanzamiento       & 1992                         \\ \hline
Lenguaje de programación & C++, Python y C              \\ \hline
Sitio Web                & \url{https://www.samba.org} 	\\ \hline
\end{tabular}
\caption{Ficha técnica Samba}
\end{table}

Samba es una implementación libre del protocolo usado para compartir archivos de Microsoft desarrollado originalmente para Unix por Andrew Tridgell utilizando técnicas de ingeniería inversa para averiguar el funcionamiento del protocolo. Aunque sigue siendo un protocolo propietario a partir de la versión 2.0 (2006) Microsoft comenzó a publicar las especificaciones del protocolo SMB para permitir la interoperatibilidad entre diferentes sistemas operativos.


\subsection {NTP - Network Time Protocol}

Network Time Protocol (NTP) es un protocolo utilizado para sincronizar los relojes de diversos sistemas informáticos.

\subsubsection {Implementaciones libres}

\begin{table}[H]
\begin{tabular}{|l|l|}
\hline
Nombre                   & OpenNTPD                       \\ \hline
Licencia                 & ISC                            \\ \hline
Año de lanzamiento       & 2004                           \\ \hline
Lenguaje de programación & C                              \\ \hline
Sitio Web                & \url{http://www.openntpd.org}  \\ \hline
\end{tabular}
\caption{Ficha técnica OpenNTPD}
\end{table}

OpenNTPD es una implementación del protocolo NTP para sincronizar el reloj del sistema contra servidores NTP remotos. También puede actuar como un servidor NTP para clientes compatibles con NTP. OpenNTPD está desarrollado principalmente por Henning Brauer como parte del proyecto OpenBSD.

\bigskip
La motivación para desarrollar OpenNTPD fue una combinación de problemas con las implementación NTP existentes como pueden ser una configuración difícil, código complejo y difícil de auditar así como licencias incompatibles con la licencia BSD.


\subsection {VPN - Virtual Private Network}

Una red privada virtual (VPN) es una tecnología que permite crear una red segura de acceso local (LAN) sobre una red pública como puede ser Internet.

\bigskip
El protocolo más utilizado es IPSEC, pero también existen protocolos como pueden ser PPTP y L2TP. Cada uno con sus ventajas y desventajas en cuanto a seguridad, facilidad, mantenimiento y tipos de clientes soportados.



